\begin{frame}{Eine einfache Funktion}
% \begin{lstlisting}
% public class Calculator {   
%    public double f ( double d ) {
%       if( d < 5.0 ) return 3.0;
%       else if( d < 10.0 ) return 2.0;
%       else return 4.0;
%    }
% }
% \end{lstlisting}

\begin{minipage}[t]{5cm}
Beispielwerte für f:

\begin{tabular}{|c|c|}
\textbf{x} & \textbf{f( x )}  \\ \hline
1 & 3  \\
3 & 3  \\
7 & 2  \\
12 & 4
\end{tabular} 
\end{minipage}
\begin{minipage}[t]{5cm}
Unit-Tests für  f:

\begin{lstlisting}
@Test
public void valuesOfF() {

assertEquals( 3.0, f( 1.0 ), 0.01 );
assertEquals( 3.0, f( 3.0 ), 0.01 );
assertEquals( 2.0, f( 7.0 ), 0.01 );
assertEquals( 4.0, f( 12.0 ), 0.01 );
}
\end{lstlisting}
\end{minipage}
\end{frame}

\begin{frame}{Was ist das Problem?}

\glqq{}Traditional test suites verify a few well-picked scenarios or example inputs. However, such example-based testing does not uncover errors in legal inputs that the test writer overlooked.\grqq{}

\hfill[Saff et.al.]

\end{frame}

\begin{frame}{Eine Alternative: Spezifikationen}
Eine geeignete mathematische Spezifikation ist z. B.:
\begin{eqnarray*}
\forall x. \; \parbox{2cm}{$(x < 5$} &\Rightarrow& f(x) = 3) \\
\land \; \parbox{2cm}{$(5 \leq x < 10$} &\Rightarrow& f(x) = 2)\\
\land \; \parbox{2cm}{$(10 \leq x$} &\Rightarrow& f(x) = 4)
\end{eqnarray*}

Kann man so etwas als Unit-Test formulieren?
\end{frame}

\begin{frame}{Eine Lösung: JUnit Theories}

\glqq{}A theory generalizes a (possibly infinite) set of example-based tests. A theory is an assertion that should be true for any data, and it can be exercised by human-chosen data or by automatic data generation.\grqq

\hfill[Saff et.al.]

\end{frame}

\begin{frame}{Theories}
	\begin{itemize}
		\item Neu seit JUnit Version 4.4
		\item Parametrisierte Testmethode mit Vorbedingungen (Assumptions)
			\begin{itemize}
				\item \texttt{assumeThat()}
				\item \texttt{assumeTrue()}
				\item \texttt{assumeNotNull()}
			\end{itemize}
		\item \texttt{@Theory} statt \texttt{@Test}
		\item Input für Testmethoden über \texttt{@DataPoint}(\texttt{s})-Annotation
	\end{itemize}
\end{frame}

\begin{frame}{Herkömmliche Tests vs. Theories}
	\begin{itemize}
		\item Herkömmliche Tests benutzen Beispiele:
			\begin{itemize}
				\item Überprüfung des Verhaltens unter ausgewählten Eingaben
				\item Es sollten charakteristische Beispiele gewählt werden
			\end{itemize}
		\item Eine Theory verallgemeinert eine Menge von Tests:
			\begin{itemize}
				\item Vorbedingung wird explizit angegeben
				\item Sollte für alle Eingaben gelten, die die Vorbedingungen erfüllen
			\end{itemize}
		\item Eingabewerte können explizit angegeben werden oder automatisch erzeugt werden
	\end{itemize}
\end{frame}


\begin{frame}{Theories für unsere Funktion}
% \begin{lstlisting}
% @Theory
% public void valuesLessThan5( double d ) {
%    assume( d < 5.0 );
%    assertEquals( 3.0, f(d), 0.01 );
% }

% @Theory
% public void valuesLessThan10( double d ) {
%    assume ( 5.0 <= d ); 
%    assume( d < 10.0 );
%    assertEquals( 2.0, f(d), 0.01 );
% }

% @Theory
% public void values10OrGreater( double d ) {
%    assume( 10.0 <= d );
%    assertEquals( 4.0, f(d), 0.01 );
% }
% \end{lstlisting}

Und woher kommen die Testwerte?

\begin{lstlisting}
@DataPoint
public static double VALUE1 = 1.0;

@DataPoint
public static double VALUE2 = 3.0;

@DataPoint
public static double VALUE3 = 7.0;

@DataPoint
public static double VALUE4 = 12.0;
\end{lstlisting}

\end{frame}



