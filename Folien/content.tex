\begin{frame}{TODO}
\end{frame}

\begin{frame}{Theories}
	\begin{itemize}
		\item Neu seit Version 4.4
		\item Parametrisierte Testmethode mit Vorbedingungen (Assumptions)
			\begin{itemize}
				\item \texttt{assumeThat()}
				\item \texttt{assumeTrue()}
				\item \texttt{assumeNotNull()}
			\end{itemize}
		\item \texttt{@Theory} statt \texttt{@Test}
		\item Input für Testmethoden über \texttt{@DataPoint}(\texttt{s})-Annotation
	\end{itemize}
\end{frame}

\begin{frame}{Herkömmliche Tests vs. Theories}
	\begin{itemize}
		\item Herkömmliche Tests benutzen Beispiele:
			\begin{itemize}
				\item Überprüfung des Verhaltens unter ausgewählten Eingaben
				\item Beispiele sind (hoffentlich) charakteristisch
			\end{itemize}
		\item Eine Theory verallgemeinert eine Menge von Tests:
			\begin{itemize}
				\item Vorbedingung wird explizit angegeben
				\item Sollte für alle Eingaben gelten, die Vorbedingungen erfüllen
			\end{itemize}
		\item Eingabewerte können explizit angegeben werden oder automatisch erzeugt werden
	\end{itemize}
\end{frame}

